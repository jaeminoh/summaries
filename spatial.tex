\section{Spatial Data Analysis}

\subsection{Hierarchichal Modeling and Analysis for Spatial Data\cite{banerjee2015}}

\begin{enumerate}
    \item Overview of spatial data problems
    
    \ Spatial data has three possible different forms: 
    point referenced data, areal data, and point pattern data.
    Let $D \subset \mathbb{R}^d$ be a set of locations.
    If the data can be described as $Y(s_i)$ where $s_i \in D$ and $s_i$ is deterministic,
    then it belongs to the class of point referenced data.
    Instead of the exact location, imagine that the information in $B_i \in 2^D$ is given.
    We call this case as an areal data.
    When $D$ is a random set, then it is a point pattern data.

    \ For point referenced data, it is natural to think that $Cov\left ( Y(s_i), Y(s_j) \right )$ 
    is a function of a distance between $s_i$ and $s_j$.
    The most convenient approach is assuming
    \[
        \begin{split}
            &(Y(s_i), \dots, Y(s_m)) \sim N_m \left ( \mu, \Sigma \right )\\
            &(\Sigma)_{ij} = \sigma^2 e^{-\phi d_{ij}^\kappa} + \tau^2 I(i=j)
        \end{split}
    \]
    where $\tau^2$ is called a \emph{nugget effect}.
\end{enumerate}